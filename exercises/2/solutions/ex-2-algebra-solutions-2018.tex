\documentclass[11pt]{article}

% font things
\usepackage{amsmath}
\usepackage{MnSymbol} % math things
% serif: if MinionPro doesn't work, use mathptmx
\usepackage[lf, mathtabular, minionint]{MinionPro} % Minion
% \usepackage{mathptmx}   % times
% sans font: use roboto if MyriadPro doesn't work
% \usepackage{MyriadPro} 
\usepackage{roboto}     % sans 


% begin old preamble

% --- character encoding ---
% \usepackage[latin1]{inputenc}
% \usepackage[T1]{fontenc}


\usepackage[top=.8in, bottom=.8in, left=1in, right=1in]{geometry}

% --- old font ---
% \renewcommand{\rmdefault}{pplx}
% \usepackage[sc]{mathpazo}
% \usepackage[OT1, euler-hat-accent]{eulervm}
\usepackage[usenames, dvipsnames, svgnames]{xcolor}
\usepackage{enumitem}

% --- styling ---
\usepackage{titling}
\usepackage[small, compact]{titlesec}
\setitemize[0]{leftmargin=*}
\usepackage{multicol, multirow}
\usepackage{epsfig, subfigure, subfloat, graphicx}
\usepackage{anysize, indentfirst, setspace}
\usepackage{verbatim, rotating, xfrac}
\usepackage{gensymb}
\usepackage{caption, hanging}
\newcommand{\mc}[1]{\multicolumn{1}{c}{#1}}
%\parindent 0pt
%\setdefaultenum{a.}{i.}{A}{1}
%\setdefaultitem{}{\textperiodcentered}{}{}
\usepackage{booktabs}
\usepackage{dcolumn}
\usepackage{caption, hanging}
\usepackage{tikz}
\usetikzlibrary{shapes,arrows,backgrounds}
%\setdefaultenum{a.}{1)}{i.}{a.}
\parindent 0pt

\makeatletter
\newcommand{\distas}[1]{\mathbin{\overset{#1}{\kern\z@\sim}}}%
\newsavebox{\mybox}\newsavebox{\mysim}
\newcommand{\distras}[1]{%
  \savebox{\mybox}{\hbox{\kern3pt$\scriptstyle#1$\kern3pt}}%
  \savebox{\mysim}{\hbox{$\sim$}}%
  \mathbin{\overset{#1}{\kern\z@\resizebox{\wd\mybox}{\ht\mysim}{$\sim$}}}%
}
\makeatother




\title{\Large{\bf{\vspace{-100pt}Mathematics for Political Science \vspace{-15pt}}}}
\author{\large{Day 1: Introduction, Foundations, Pre-Calculus}}
\date{\vspace{-5pt}\large{Solutions \vspace{-10pt}}}
\begin{document}


\maketitle

\hrule




\begin{enumerate}



\item 
\begin{enumerate}
\item $x = \frac{1}{3}$
\item $x = \frac{3}{4}$
\end{enumerate}


\item 
\begin{enumerate}
\item $\alpha = \beta + 4\theta$
\item $\alpha = \frac{4}{(x + y - x^2 - y^2)}$
\end{enumerate}


\item 
\begin{enumerate}
\item $x > -18$
\item $t < 6$
\item $y \leq \frac{29}{22}$
\end{enumerate}


\item 
\begin{enumerate}
\item $x = 2$ or $x=-7$
\item $x=4$
\item $x=2$ or $x=-5$
\end{enumerate}


\item 
\begin{enumerate}
\item $x=\frac{1}{9}$ or $x=-1.5$
\item $x= -\frac{2}{7}$ or $x=\frac{4}{5}$
\end{enumerate}


\item 
\begin{enumerate}
\item $a=0$, $b=2$
\item $a=5$, $b=5$
\end{enumerate}



\item 
\begin{enumerate}
\item $c=7$, $d=-2$
\item $c=-3$, $d=4$
\end{enumerate}


\item $x=4\alpha + 2$, $y=2\alpha + 1$


\item $q=1$, $r=-1$, $s=3$


\item 
\begin{enumerate}
\item $25$
\item $22$
\end{enumerate}




\item Odd powers are identical to the matrix given; even powers are the identity matrix.


\item 
\[
\left[\begin{array}{cc}
a & b \\
c & d \\
\end{array}\right]
+
\left[\begin{array}{cc}
0 & 0 \\
0 & 0 \\
\end{array}\right]
=
\left[\begin{array}{cc}
a+0 & b+0 \\
c+0 & d+0 \\
\end{array}\right]
=
\left[\begin{array}{cc}
a & b \\
c & d \\
\end{array}\right]
\]

\[
\left[\begin{array}{cc}
a & b \\
c & d \\
\end{array}\right]
\left[\begin{array}{cc}
1 & 0 \\
0 & 1 \\
\end{array}\right]
=
\left[\begin{array}{cc}
a*1 + b*0 & a*0 + b*1 \\
c*1 + d*0 & c*0 + d*1 \\
\end{array}\right]
=
\left[\begin{array}{cc}
a & b \\
c & d \\
\end{array}\right]
\]


\item 
\begin{enumerate}
\item $\left[\begin{array}{cc}
56 & 70 \\
\end{array}\right]$
% [56  70]
\item $
\left[\begin{array}{c}
ap + bq + cr \\
dp + eq + fr \\
gp + hq + ir \\
\end{array}\right]$
\item The inner dimensions do not conform, so these matrices cannot be multiplied in this order.
\end{enumerate}




\item Multiply the matrices below to show that order matters for matrix multiplication: \\

\begin{enumerate}
\item $\left[\begin{array}{c}
17 \\
\end{array}\right]
~~~\textrm{or}~~~
\left[\begin{array}{ccc}
12 & 21 & 3 \\
0  & 0  & 0 \\
20 & 35 & 5 \\
\end{array}\right]$
\item $\left[\begin{array}{ccc}
44 & 64 & 36 \\
15 & 36 & 21 \\
20 & 22 & 12 \\
\end{array}\right]
~~~\textrm{or}~~~
\left[\begin{array}{cc}
48 & 114 \\
15 & 44 \\
\end{array}\right]$
\end{enumerate}



\end{enumerate}




\vfill
\begin{center}
\small{Thanks to Dave Ohls, Brad Jones, and Sarah Bouchat for past years' materials}
\end{center}

\end{document} 